%! Author = stephane
%! Date = 18.02.22

% Preamble
\documentclass[10pt,a4paper]{recettes}

% Packages
\usepackage[utf8]{inputenc} % Encodage d'entrée
\usepackage[frenchb]{babel} % Langage
\usepackage[T1]{fontenc} % Encodage de sortie
\usepackage{graphicx} % Extension pour les images
\usepackage[left=2cm, right=2cm, top=2cm, bottom=2cm]{geometry} % marges

\usepackage{blindtext}
\usepackage{multicol}
\setlength{\columnsep}{15pt}
\usepackage{color}
\setlength{\columnseprule}{1pt}
\def\columnseprulecolor{\color{black}}

% Document
\begin{document}

    % Affichage du titre
    \maketitle

    % Affichage de la table des matières
    \tableofcontents

    \section{Omelette espagnole}

    \begin{multicols}{2}
        \parbox[1cm]{\textwidth}{
            \begin{ingredients}
                \item \textit{3 cuillère à soupe d'huile d'olive}
                \item \textit{1 à 2 oignons hachés}
                \item \textit{2 gousses d'ail écrasées}
                \item \textit{1 poivron rouge évidé et haché}
                \item \textit{4 oeufs}
                \item \textit{sel et poivre}
                \item \textit{2 grosses pommes de terre cuites et émincées}
                \item \textit{2 cuillères à soupe de persil haché}
            \end{ingredients}
        }

        \columnbreak
        Faites chauffer 2 cuillère à soupe d'huile dans une poêle de 24 cm, ajoutez et faites revenir les oignons. Ajouter l'ail, le poivron et laissez cuire 10 minutes.
        \newline

        Battez les oeufs dans une jatte avec le sel, le poivre, puis ajoutez-leur les pommes de terre, le persil et le mélange précédent.
        \newline

        Faites chauffer le reste d'huil dans la poêle, versez le mélange et laissez cuire 5 minutes, en secouant la poêle.
        \newline
        \newline
        \textbf{Pour 4 personnes}
    \end{multicols}
\end{document}