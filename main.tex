%! Author = stephane
%! Date = 18.02.22

% Preamble
\documentclass[10pt,a4paper]{recettes}

% Packages
\usepackage[utf8]{inputenc} % Encodage d'entrée
\usepackage[frenchb]{babel} % Langage
\usepackage[T1]{fontenc} % Encodage de sortie
\usepackage{graphicx} % Extension pour les images
\usepackage[left=2cm, right=2cm, top=2cm, bottom=2cm]{geometry} % marges


% Document
\begin{document}

    % Affichage du titre
    \maketitle

    % Affichage de la table des matières
    \tableofcontents

    \section{Omelette espagnole}
    3 cuillère à soupe d'huile d'olive
    2 oignons hachés
    2 gousses d'ail écrasées
    1 poivron rouge évidé et haché
    4 oeufs
    sel et poivre
    2 grosses pommes de terre cuites et émincées
    2 cuillères à soupe de persil haché

    Faites chauffer 2 cuillère à soupe d'huile dans une poêl de 24 cm, ajoutez et faites revenir les oignons. Ajouter l'ail, le poivron et laissez cuire 10 minutes.
    Battez les oeufs dans une jatte avec le sel, le poivre, puis ajoutez-leur les pommes de terre, le persil et le mélange précédent.
    Faites chauffer le reste d'huil dans la poêle, versez le mélange et laissez cuire 5 minutes, en secouant la poêle.
    Passez 3 minutes sous le gril, faites glisser l'omelette sur le plat de service et coupez-la en portions.

    Pour 4 personnes
\end{document}